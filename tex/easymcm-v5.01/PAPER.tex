%%%%%%%%%%%%%%%%%%%%%%%%%%%%%%
% 	   美赛模板,正文部分		 %
%          PAPER.tex         %
%%%%%%%%%%%%%%%%%%%%%%%%%%%%%%

\documentclass[12pt]{article}

% 请在此填写控制号、题号和标题 
% 年份不需要填(自动以当前电脑时间年份为准)
\usepackage[12345]{easymcm}\problem{A}   
\usepackage{palatino} % 这个是COMAP官方杂志的字体,如不需要可注释掉,以使用默认字体
\title{An MCM Paper Made by Team 12345}  % 标题

% 正文开始
\begin{document}
%%%%%%%%%%%%%%%%%%%%%%%%%%%%%%%%%%%%%%%%%
%%            请在此填写摘要            %%
%%%%%%%%%%%%%%%%%%%%%%%%%%%%%%%%%%%%%%%%%
\begin{abstract}\small
A non-SPA-style bathtub cannot be reheated by itself, so the water will get noticeably
cooler and users should add hot water from time to time. Based on the existing Partial
Differential Equation (PDE) technique, we construct time-space based temperature model
which can simulate any common condition. And then propose an optimal strategy
for users to keep the temperature even and close to initial temperature and decrease the
water consumption.

Firstly, we construct the PDE-solving model to simulate the temperature distribution
in the bathtub by analyzing the heat loss and confirming the corresponding parameters.
In this part, we consider two kinds of heat transfer, (water-bathtub heat conduction,
water-air heat convection), and the effect of inflow water on the faucet is shown by the
heat source. So we can confirm the boundary conditions of PDE to carry out the next
step.

Secondly, we do parameter testing by free cooling process to ensure the parameters
are in line with the actual condition. Most of the parameters are applicable, and a few
parameters are fine-tuned to make the model more accurate and practicable.
Thirdly, we confirm the optimal strategy by analyzing and comparing the continuous
and discontinuous flow methods, which vary from the temperature and flux. There are
two feasible methods gained from the analyses, one is continuous 42 $^\circ$C water inflow, the
other is turning on and off the faucet with 42 $^\circ$C inflow for 10 minutes.In addition, we
give two more kinds of methods.

At last, we use our model to determine the extent to which our strategy depends on
the factors of the bathtub, the factors of the user, and a bubble bath additive. And we test
the methods by changing the initial temperature. From those results, we further suggest
the users move less and use more bubble bath additive, and turning on the faucet at the
beginning is also a good choice.
\end{abstract}



%%%%%%%%%%%%%%%%%%%%%%%%%%%%%%%%%%%%%%%%%%
% 如不理解以下部分中各命令的含义,请勿修改! %
%%%%%%%%%%%%%%%%%%%%%%%%%%%%%%%%%%%%%%%%%%

%---------以下生成sheet页----------
% 下面的语句可调整全文行距为标准值的0.6倍,请自行使用
%\renewcommand{\baselinestretch}{0.6}\normalsize
\maketitle  % 生成sheet页
\thispagestyle{empty}   % 不要页眉页脚和页码
\setcounter{page}{-100} % 此命令仅是为了避免页码重复报错,不要在意

%---------以下生成目录----------
\newpage
\tableofcontents
\thispagestyle{empty}   % 不要页眉页脚和页码
\newpage

%---------以下生成正文----------
\setlength\parskip{0.8\baselineskip}  % 调整段间距
\setcounter{page}{1}    % 从正文开始计页码
\pagestyle{fancy}		% 摘要请到ABSTRACT.tex中填写

\section{Introduction}
\subsection{Problem Background}
Here is the problem background ...

Two major problems are discussed in this paper, which are:
\begin{itemize}
    \item Doing the first thing.
    \item Doing the second thing.
\end{itemize}

\subsection{Literature Review}
A literatrue\cite{1} say something about this problem ...

\subsection{Our work}
We do such things ...

\begin{enumerate}[\bfseries 1.]
    \item We do ...
    \item We do ...
    \item We do ...
\end{enumerate}

\section{Preparation of the Models}
\subsection{Assumptions}

\subsection{Notations}
The primary notations used in this paper are listed in \textbf{Table \ref{tb:notation}}.
\begin{table}[!htbp]
\begin{center}
\caption{Notations}
\begin{tabular}{cl}
	\toprule
	\multicolumn{1}{m{3cm}}{\centering Symbol}
	&\multicolumn{1}{m{8cm}}{\centering Definition}\\
	\midrule
	$A$&the first one\\
	$b$&the second one\\
	$\alpha$ &the last one\\
	\bottomrule
\end{tabular}\label{tb:notation}
\end{center}
\end{table}

\section{The Models}
\subsection{Model 1}
\subsubsection{Detail 1 about Model 1}
\begin{equation}
    e^{i\theta}=\cos\theta+i\sin\theta.
\end{equation}

\section{Strengths and Weaknesses}
\subsection{Strengths}
\begin{itemize}
    \item First one...
    \item Second one ...
\end{itemize}

\subsection{Weaknesses}
\begin{itemize}
    \item Only one ...
 \end{itemize}

\begin{thebibliography}{99}
\addcontentsline{toc}{section}{References}  %引用部分标题("Refenrence")的重命名
\bibitem{1}Elisa T. Lee, Oscar T. Survival Analysis in Public Health Research. \emph{Go.College of Public Health}, 1997(18):105-134.
\bibitem{2}Wikipedia: Proportional hazards model. 2017.11.26. \texttt{\\https://en.wikipedia.org/wiki/Proportional\_{}hazards\_{}model}
\end{thebibliography}

\end{document}
